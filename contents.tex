%!TEX root = ./Assignment.tex

\begin{figure}[h]
    \begin{center}
        \includegraphics[width=0.9\textwidth]{Figures/1}
    \end{center}
\caption{Something}
\end{figure}


\begin{cli}
admin@system:~$ ls -lh | grep "something"
admin@system:~$ ls -lh
admin@system:~$ 
\end{cli}


\begin{quotebox}{\bccrayon}

    Something nice

\end{quotebox}

\begin{itemize}

    \item Stuff

        \begin{itemize}
        
            \item more stuff
            
        \end{itemize}
    
\end{itemize}

\begin{enumerate}

    \item Stuff

        \begin{itemize}
        
            \item More stuff
            
        \end{itemize}

\end{enumerate}

\begin{step}{Date \& Time}

    \underline{There are no classes on Monday, October 9}; thus, it is not possible to conduct the test during the lecture. However, your lab instructor will conduct the \textbf{midterm test during the lab hours between October 10 and 13}. 

    \textbf{Notice:} If I am your lab instructor on Monday, then you will take the midterm during the lab time on Monday October 16.

\end{step}

\begin{step}{Duration}

    You will have a total of \textbf{60 minutes} to \underline{complete and submit} the test.

\end{step}

\begin{step}{Format}

    \begin{itemize}

        \item The test will be a combination of MCQ and Short Answer Questions

        \begin{itemize}
        
        \item \textbf{15 MCQs} [50\%] 

        \item \textbf{5 short answer questions} [50\%] 
            
        \end{itemize}
        
        \item You can have access to your \textbf{Linux Server, \underline{not the desktop version.}}

        \item You will take the test through \textbf{Microsoft Forms} using the school's computers.

    \end{itemize}
    
\end{step}

\begin{step}{Contents}
    
    The test will \underline{comprehensively} evaluate knowledge from \textbf{weeks 1 to 5}.

\end{step}


\begin{step}{Rules}
    
    \begin{itemize}

        \item \textbf{Absolutely no communication during the test.}

        \item \textbf{Cheating will result in a grade of zero in the test and an Academic Misconduct Form.}

    \end{itemize}

\end{step}

\begin{step}[colframe=odblue]{Sample MCQ}
    
\textbf{Question:} Which of the following Linux distribution is best suited for servers?

        \begin{itemize}
            \item Ubuntu Desktop
            \item CentOS
            \item Kali Linux
            \item Manjaro
        \end{itemize}

\textbf{Answer: CentOS}

\end{step}

\begin{step}[colframe=odblue]{Sample Short Answer Question}
    
\textbf{Question:} See my prompt below. I have a file named \inlineCLI{file1} in my home directory. Type the command to append the string "I am acing this test" to \inlineCLI{file1} without changing the current working directory.

\begin{cli}
# My prompt is as below
admin@system:/usr$ 
\end{cli}

\textbf{Possible Answer:} \inlineCLI{echo "I am acing this test" >> /home/admin/file1}\\

\textbf{Possible Incorrect Answer:} \inlineCLI{echo "I am acing this test" >> file1}\\

\textbf{Explanation:} The promt lets me know that my current working directory is \inlineCLI{/usr}. Thus, if I want to append a string to \inlineCLI{file1}, I need to specify the file's absolute path after stdout redirection.

\end{step} 

\begin{center}

\resizebox{0.9\linewidth}{!}{%

\begin{tikzpicture}[node distance = 2cm, every node/.style={font=\bfseries}]

  \node (start) [startstop] {START};
  \node (input) [io, below of=start, yshift=-1.4cm] {INPUT: $\symbf{a}$, $\symbf{b}$};
  \node (decision) [decision, below of=input, yshift=-2cm] {Is\\ $\symbf{a > b}$\\?};
  \node (comment) [comment, right of=decision, xshift=5cm, font=\small\itshape] {\,\,We use the diamond symbol to represent\\ decisions, like an "IF"};
  \node (truebranch) [process, left of=decision, xshift=-5cm] {$\symbf{g \longleftarrow a}$};
  \node (falsebranch) [process, below of=decision, yshift=-1.5cm] {$\symbf{g \longleftarrow b}$};
  \node (output) [io, below of=falsebranch, yshift=-0.5cm] {OUTPUT: $\symbf{g}$};
  \node (stop) [startstop, below of=output, yshift=-1cm] {STOP};
  
  % Connect nodes with arrows
  \draw [arrow] (start) -- (input);
  \draw [arrow] (input) -- (decision);
  \draw [line] (decision) -- (comment);
  \draw [arrow] (decision) -- node[anchor=south] {true} (truebranch);
  \draw [arrow] (decision) -- node[anchor=east] {false} (falsebranch);
  \draw [arrow] (truebranch) |- (output);
  \draw [arrow] (falsebranch) -- (output);
  \draw [arrow] (output) -- (stop);

\end{tikzpicture}

}

\end{center}

\lipsum[1-7]

\begin{step}{Instructions \& Specs}

    \textbf{Instructions:}

    \begin{enumerate}

        \item Create a new Word document. 

        \item Include a cover page with relevant information:

            \begin{itemize}
            
                \item Your name
                
                \item Your student ID
                
                \item Your section
                
                \item The course name, i.e., Introduction to Programming with PowerShell.
                
            \end{itemize}

        \item Take screenshots where specified in this lab and paste them into your Word document \textcolor{NordOrange}{\textbf{in the correct order}}. 

        \item Answer \textcolor{NordOrange}{\textbf{all}} the questions (if any.)

        \item Save the Word file as \textcolor{NordYellow}{\texttt{Assignment\_8\_FirstName\_LastName.docx}}. Also, save your code as \textcolor{NordYellow}{\texttt{Assignment\_8\_FirstName\_LastName.txt}} file.

        \item Upload your word and .txt files to the lab on Teams.
            
    \end{enumerate}

    \textbf{Specifications:}

    \begin{enumerate}

        \item Ensure each figure shows all the required information, including information that certifies the authenticity of your work, for example:

            \begin{itemize}
            
            \item \underline{Comments in your code} 
            
            \item \underline{Unique Solutions}
            
            \item \underline{Your name and ID in the places I specify}
                
            \end{itemize}
            
        \item Use unique formatting so your work is easily distinguishable from others, but remember that \textbf{\textcolor{NordOrange}{your document must look professional};} please refrain from using fluorescent colours and extravagant fonts. 

        \item Total number of screenshots and questions: \textbf{\textcolor{NordRed}{1}}.

    \end{enumerate}
    
\end{step}

\section{Procedure}

\subsection{Problems}

\begin{step}[colframe=odgreen]{Problem}

    Create a PowerShell script that does the following:

    \begin{itemize}
    
        \item Use a while loop to populate an array with the user's input. The elements of the array should be positive integers.

        \item Create a function that calculates the \href{https://en.wikipedia.org/wiki/Geometric_mean}{\color{NordGreen}{\underline{Geometric Mean}}}  of the numbers above.

        \item The function should receive pipeline input from the array you created above. 

        \item Ensure to include parameter validation so the elements of the array are not larger than 10000.

        \item Ensure the user's values do not produce an indefinite result, i.e., calculate an even root of a negative number.
        
    \end{itemize}

\end{step}

